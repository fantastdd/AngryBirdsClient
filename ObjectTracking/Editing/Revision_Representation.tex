
\documentclass[letterpaper]{article}
\usepackage{aaai}
\usepackage{times}
\usepackage{helvet}
\usepackage{algorithm}
\usepackage{algpseudocode}
\usepackage{courier}
\usepackage{graphicx}
\graphicspath{{./figure/}}
\frenchspacing
\setlength{\pdfpagewidth}{8.5in}
\setlength{\pdfpageheight}{11in}
\begin{document}    

\section{Objects Representation with the extended GSR-n (EGSR)} 

We use a minimum bounding rectangle (MBR) to approximate the region occupied by a circle and use exact shapes for the general solid rectangles (GSR), i.e. rectangles that can have any angle and impenetrable. 

Many rectangle-based spatial calculi \cite{balbiani1998model,cohn2012thinking,sokeh2013efficient} are developed in the context of Qualitative Spatial Reasoning for dealing with different spatial aspects of such as topology, size etc. One of them is GSR-n, proposed by \cite{Ge2013}. GSR-n is a comprehensive spatial representation of GSRs. It defines eight contact sectors that correspond to the eight edges and corners of the rectangles. Given two GSRs $o_1$ and $o_2$ that contact each other via $sector_1, sector_2 \in \{S_1, ..., S_8, R_1, ..., R_8\}$, the contact relation between $o_1$ and $o_2$ can be expressed as the constraint $o_1 \, (sector_1, sector_2) \, o_2$ (Figure \ref{GSR}). With the contact relations, GSR-n allows us to distinguish if and how two objects contact each other. Since the objects can only interact via contacts, we can further infer the possible motions of an object from the GSR relations the object holds with others. GSR-n also defines a set $n$ of unary relations by the $Qualitative\,Corner\,Instantiations$, which qualitatively describes the size and leaning direction of a GSR. Since we represent the rectangles using real shapes, we do not need the unary relations. 
\begin{figure}[h!]
\centering\includegraphics[scale=0.25]{GSR.png}\caption{Contact sectors of (a) a normal rectangle (without rotation) and (b) an angular rectangle. (c) An example scenario where $o_1 (R_7, S_4) o_2$, $o_2 (S_8, S_4) o_3$}
\label{GSR}
\end{figure}

Given two GSRs, we obtain the contact relation by enumerating all the plausible combinations of the two GSRs' contact sectors and for each combination calculating the distance between the two sectors. The combination with the shortest distance constitutes the contact relation. Note, the shortest distance can be non-zero. A non-zero distance means the two GSRs are separate, otherwise touch. So unlike the original definition that the contact relation can only be assigned when two GSRs touch, here we can assign a contact relation for any pair of GSRs. To distinguish whether two GSRs touch or not, we turn to qualitative distance which is covered in the next section.

The problem with GSR-n is that it uses $(\emptyset, \emptyset)$ to represent the spatial relation between all the non-touched GSRs. Thus, it introduces ambiguities when the rectangles are disconnected. To overcome the ambiguities, we extend the original GSR-n by integrating it with the cardinal tiles \cite{goyal1997direction}. Specifically, we partition the embedding space around an reference object into nine mutually exclusive tiles . The centre tile corresponds to the minimum bounding rectangle of the object and the other eight tiles correspond to the eight cardinal directions. We call this new representation EGSR (see Figure \ref{EGSR}). All EGSR relations are obviously converse, e.g. the converse of $TOP\,LEFT$ is $BOTTOM\,RIGHT$, the converse of $(R_4, S_1)$ is $(S_1,R_4$).

We compute the EGSR relation between two spatial objects by first checking whether their MBRs intersect or boundary touch. If so, we assign a GSR-n relation accordingly. Otherwise, one of the eight cardinal tiles will be used. 

\section{Objects Representation with the extended GSR-n (EGSR)} 

We use a minimum bounding rectangle (MBR) to approximate the region occupied by a circle and use exact shapes for the general solid rectangles (GSR), i.e. rectangles that can have any angle and impenetrable. 

Many rectangle-based spatial calculi \cite{balbiani1998model,cohn2012thinking,sokeh2013efficient} are developed in the context of Qualitative Spatial Reasoning for dealing with different spatial aspects of such as topology, size etc. One of them is GSR-n, proposed by \cite{Ge2013}. GSR-n is a comprehensive spatial representation of GSRs. It defines eight contact sectors that correspond to the eight edges and corners of the rectangles. Given two GSRs $o_1$ and $o_2$ that contact each other via $sector_1, sector_2 \in \{S_1, ..., S_8, R_1, ..., R_8\}$, the contact relation between $o_1$ and $o_2$ can be expressed as the constraint $o_1 \, (sector_1, sector_2) \, o_2$ (Figure \ref{GSR}). With the contact relations, GSR-n allows us to distinguish if and how two objects contact each other. Since the objects can only interact via contacts, we can further infer the possible motions of an object from the GSR relations the object holds with others. GSR-n also defines a set $n$ of unary relations by the $Qualitative\,Corner\,Instantiations$, which qualitatively describes the size and leaning direction of a GSR. Since we represent the rectangles using real shapes, we do not need the unary relations. 
\begin{figure}[h!]
\centering\includegraphics[scale=0.25]{GSR.png}\caption{Contact sectors of (a) a normal rectangle (without rotation) and (b) an angular rectangle. (c) An example scenario where $o_1 (R_7, S_4) o_2$, $o_2 (S_8, S_4) o_3$}
\label{GSR}
\end{figure}

Given two GSRs, we obtain the contact relation by enumerating all the plausible combinations of the two GSRs' contact sectors and for each combination calculating the distance between the two sectors. The combination with the shortest distance constitutes the contact relation. Note, the shortest distance can be non-zero. A non-zero distance means the two GSRs are separate, otherwise touch. So unlike the original definition that the contact relation can only be assigned when two GSRs touch, here we can assign a contact relation for any pair of GSRs. To distinguish whether two GSRs touch or not, we turn to qualitative distance which is covered in the next section.

The problem with GSR-n is that it uses $(\emptyset, \emptyset)$ to represent the spatial relation between all the non-touched GSRs. Thus, it introduces ambiguities when the rectangles are disconnected. To overcome the ambiguities, we extend the original GSR-n by integrating it with the cardinal tiles \cite{goyal1997direction}. Specifically, we partition the embedding space around an reference object into nine mutually exclusive tiles . The centre tile corresponds to the minimum bounding rectangle of the object and the other eight tiles correspond to the eight cardinal directions. We call this new representation EGSR (see Figure \ref{EGSR}). All EGSR relations are obviously converse, e.g. the converse of $TOP\,LEFT$ is $BOTTOM\,RIGHT$, the converse of $(R_4, S_1)$ is $(S_1,R_4$).

We compute the EGSR relation between two spatial objects by first checking whether their MBRs intersect or boundary touch. If so, we assign a GSR-n relation accordingly. Otherwise, one of the eight cardinal tiles will be used. 

\begin{figure}[h!]
\centering\includegraphics[scale=0.25]{EGSR-relations.png}\caption{Contact sectors and Cardinal directions of (a) an angular rectangle and (b) a normal rectangle}
\label{EGSR}
\end{figure}

 \begin{figure}[h!]
\centering\includegraphics[scale=0.45]{QCN.png}\caption{(a) A spatial scenario where the four rectangles form a stable structure under downward gravity (b) The corresponding qualitative constraint network}
\label{QCN}
\end{figure}
\end{document}