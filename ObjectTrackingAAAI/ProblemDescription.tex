\documentclass[letterpaper]{article}
\begin{document}
\section{Problem Description}

Let \cal{S(scene, identities)} be a state that has two components, scene and identities. Scene is every object that can be seen while identity uniquely identifies an object.    


Let \cal{B}(\cal{S},\cal{A}) be a deterministic black box that takes a state \cal{S}, an action \cal{A} as input, and outputs a partially observable state \cal{E(e, *)} where identity information is hidden. 

In order to interact with the black box (make reasonable actions), we should first learn its behaviours, specifically, what is the outcome by applying a certain action on a state. It is important to understand the changes / difference between the before and after states. However, since the black box does not provide the identity information, we are unable to capture the changes directly.    

To solve this, our task is to create a model \cal{M(\cal{S}, \cal{A}, e)} that given a state \cal{S} , an action \cal{A}, and the resulting scene \cal{e}, the model outputs the identity information for \cal{e}. By having the identify information, we can then discover the state changes, the result of an action, which is a step stone for coming up reasonable actions. 

\end{document}